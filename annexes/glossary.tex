\chapter{Acronymes et Glossaire}\label{glo}
\begin{description}
\item[API] Application Programming Interface, ensemble normalisé de classes, de méthodes ou de fonctions qui sert de façade par laquelle un
	logiciel offre des services à d'autres logiciels.
	\item[Analyseur logique] L'analyseur logique est un outil de mesure permettant de connaître au fil du temps l'évolution binaire des signaux (0 et 1) sur plusieurs voies logiques : bus de données, entrées-sorties d'un microcontrôleur ou d'un microprocesseur.
\item[Antlr] \textit{Another Tool for Language Recognition}, outil permettant de faciliter l'interprétation d'une chaîne de caractère, celui-ci prend en entrée une
	grammaire, et génère un arbre syntaxique dans plusieurs langages.
\item[Calibration] Valeur stockée en flash pouvant contenir une information permettant de simplifier la configuration véhicule. Une
	calibration pourrait être le nombre d'injecteurs.
\item[CAN] \textit{Controller Area Network}, un bus système série très répandu dans beaucoup d'industries, notamment l'automobile. Ce bus permet de racorder à un même bus un grand nombre de calculateurs qui communiqueront à tour de rôle.
\item[ControlDesk] Outil permettant de piloter le HIL, l'interface permet ainsi de modifier des valeurs de l'environnement véhicule, ou de
	pouvoir les lire graphiquement.
\item[Device] Les différents équipements dont pourrait avoir besoin l'utilisateur : Hil, Debugger, \ldots 
\item[DSpace] Société allemande fournissant Continental en simulateur d'environnement HiL, associé à une interface ControlDesk.
\item[ECU] Electronic Control Unit, calculateur du contrôle moteur.
\item[Excel] Logiciel tableur appartenant à la suite de Microsoft Office\textregistered. Il est possible de modifier une feuille de calcul depuis un logiciel ou un script, notamment en Java. 
\item[Flash] La mémoire flash est une mémoire de masse non volatile et réinscriptible. Ainsi les données sont conservées même si l'alimentation est coupée.
\item[Flasher] Action d'écrire sur la flash, dans notre cas il s'agit d'écrire ou de mettre à jour le logiciel présent sur la mémoire flash de l'ECU.
\item[Grammaire] Formalisme permettant de définir une syntaxe clair et non ambigüe.
\item[HiL] \textit{Hardware in the loop}, permet de simuler un environnement véhicule autour du calculateur du contrôleur moteur : celui-ci réagira comme s'il était embarqué dans une voiture.
\item[JAR] Java ARchive est un fichier ZIP utilisé pour distribuer un ensemble de classes Java.
\item[Java] Langage de programmation orienté Objet soutenu par Oracle. Les exécutables Java fonctionnent sur une machine virtuelle Java et permettent d'avoir un
	code qui soit portable peut importe l'hôte.
\item[JSON] JavaScript Object Notation est un format de données textuelles, générique, dérivé de la notation du langage JavaScript, il permet de représenter de
	l'information structurée.
\item[JVM] \textit{Java Virtual Machine}.
\item[Lauterbach] Société allemande spécialisée dans les outils de debuggage de systèmes embarqués. Nous utilisons leur debugger, ainsi que leur analyseur logique.
\item[Logiciel de versionnement] Logiciel, tel que \textit{Git}, permettant de maintenir facilement toutes les versions d'un logiciel, mais aussi facilitant le
	travail collaboratif.
\item[Parsing] Processus d'analyser de chaîne de caractère, en supposant que la chaîne respecte un certain formalisme. 
\item[PowerProbe] Analyseur logique de la suite Trace32 développé par la société Lauterbach.
\item[Python] Langage intérprété de programmation multi-plateforme et multi-paradigme. Il est doté d'un typage dynamique fort, d'une gestion automatique de la mémoire et d'un système de gestion d'exceptions. 
\item[Release] Version d'un logiciel qui correspond à un état donné de l'évolution d'un produit logiciel utilisant le versionnage. Ainsi, chez Continental, un projet comporte une multitude de versions différentes.
\item[TestPlan] Document détaillant les objectifs et la manière de tester une version d'un logiciel sous tests.
\item[Apache Thrift] Langage de définition d'interface conçu pour la création et la définition de services pour de nombreux langages. Il est ainsi possible de
	faire communiquer deux problèmes dans deux langages différents : Python et Java dans notre cas.
\item[Trace32] Logiciel permettant de communiquer avec le debugger JTag, ainsi qu'avec les équipements branchés en série, notamment le PowerProbe. 
%\item[Trace32] Debugger, permet de debugger un programme embarqué, ceci en permettant de lire la mémoire, mettant des points d'arrêts, \ldots
\item[UML] Unified Modeling Language est un langage de modélisation graphique. Il est utilisé en développement logiciel et en conception orienté Objet afin de
	représenter facilement un problème et sa solution.
\item[XML] Extensible Markup Language est un langage de balisage générique permettant de stocker des données textuelles sous forme d'information structurée.
\end{description}

