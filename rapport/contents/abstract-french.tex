\vspace{-45px}
\footnotesize{\textbf{Mots-clés}: Automobile, Outil, ECU, Contrôle moteur, Tests, Banc de tests, HiL, Debugger, Java, Python}

\normalsize
Dans le cadre de ma formation en seconde année de Master Développement Logiciel à l'université
Toulouse III, j'ai eu la chance de pouvoir effectuer un an d'alternance. J'ai eu l'opportunité de continuer un projet commencé précédemment : le développement
d'un outil de tests de logiciels embarqués.

L'entreprise Continental est une Société Allemande leader de l'automobile possédant plus de $170\;000$
employés dans le monde. L'entreprise s'occupe aussi bien des calculateurs que de la sécurité automobile,
du système d'injection, \ldots\newline
Pour ma part j'ai travaillé au sein de l'équipe en charge de la mise en place de services de tests.
Notamment des tests logiciels, ceci en développant des scripts de tests automatique de non-régression
ou d'intégration avant la livraison des projets.

Trois ans avant ce stage, un besoin a été exprimé : pouvoir tester de façon rapide et efficace
l'intégration d'un « plugin », un bout de code sous forme binaire, au sein des applicatifs d'un calculateur de contrôle moteur. La mission de l'équipe \textit{Tests \& Automation Service} est de permettre
de tester la bonne intégration de ce plugin avec les logiciels Continental. Pour cela le développement
d'un outil de tests est nécessaire.

Au début de mon alternance, cet outil appelée GreenT, avait déjà était bien avancée, en partie lors de mes stage
précédent. Cependant, à mon arrivée un module devait être refondu afin de pouvoir livrer une première version de l'outil. Une fois celle-ci livrée, de nouvelles
fonctionnalités ainsi que du support de la maintenance étaient nécessaires. C'est donc dans ce contexte que j'ai
travaillé durant cette année, ceci afin d'améliorer au maximum l'outil, tout en accompagnant les utilisateurs.

Cet outil a pour but d'avoir des tests les plus exhaustifs et efficaces possibles. Afin de pouvoir
tester la bonne intégration du plugin, le client fourni un fichier Excel appelé Walkthrough contenant
la liste des variables du plugin avec toutes leur spécifications. Le testeur va ajouter des colonnes à ce
fichier afin de spécifier le fonctionnement du test, notre plateforme sera ensuite capable d'analyser
le fichier, et de générer les cas de tests qui s'exécuteront à distance sur un ou plusieurs bancs de
tests : ils simulent un environnement véhicule autour du contrôleur afin de vérifier ses réactions en
fonction des différentes conditions qui peuvent arriver.

%J'ai participé à l'ajout de deux nouvelles fonctionnalités : la possibilité de modifier des calibrations 1
%en début de stimulation, fonctionnalité indispensable à l'implémentation de certains tests. J'ai
%également développé un système permettant d'améliorer la généricité des tests, en autorisant l'utilisation
%de calibrations en tant qu'index de tableau, tel que tab[calibration].

Mon travail durant ce stage aura été bénéfique, pour l'entreprise grâce à mes connaissances de l'outil, 
au développement de nouvelles fonctionnalités, à mes corrections de bugs et au support que j'ai effectué. Mais aussi
personnellement, d'un point de vue technique, en trouvant des solutions à des problèmes. Et d'un
point de vue humain grâce au travail en équipe, aux comptes rendus réguliers qui m'ont permis
d'apprendre à synthétiser mon travail.

%Le projet n'est pas terminé, celui-ci est actuellement utilisé principalement pour des projets Ford, mais celui-ci doit être utilisé par les projets Renault prochainement. 
Le projet n'est pas terminé, celui-ci nécessitant encore de la maintenance et du développement de nouvelles fonctionnalités. Cependant, avec ces deux ans de travail, j'aurai eu l'opportunité de voir un cycle logiciel complet, allant de l'analyse des besoins jusqu'à l'exploitation de l'outil, ce qui sera un atout pour ma future carrière professionnelle.
%Le projet n'est pas terminé, et je vais continuer son développement dès Septembre en contrat de
%professionnalisation avec Continental, cette expérience supplémentaire va me permettre de connaître
%un cycle logiciel complet, ce qui sera un atout pour ma future carrière professionnelle.
