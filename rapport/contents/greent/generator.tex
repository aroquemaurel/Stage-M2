	\section{Le générateur}
	Comme montré dans le schéma \ref{fig:generalDig}, une fois le walkthrough parsé, il faut générer les fichiers nécessaire au test. Pour cela, j'ai créé un \textit{package} de classes ayant des services permettant d'ajouter les fonctionnalités à générer : Ajouter un check, ajouter une affectation, ajouter une action debugger, ajouter les informations des tests, \ldots et lancer la génération du fichier.

		\subsection{Génération des tests}
		Je dois générer 3 types de fichiers : un précondstim, des stimscenarios et un \texttt{GreenTTest} pour chaque test. Chacune des classes que je génère hérite d'une classe présente dans \textit{GreenT}, respectivement \texttt{PrecondStim}, \texttt{StimScenario} et \texttt{GreenTTest}.

		La génération des fichiers possède des points communs en fonction du type de fichier, principalement entre un \texttt{PrecondStim} et un \texttt{StimScenario}, ainsi j'ai conçu un arbre d'héritage assez simple me permettant de factoriser le code : 
		\begin{figure}[H]
		\centering
		\includegraphics[width=14cm]{contents/images/generatorClass.png}
		\caption{Diagramme de classes du générateur}
		\end{figure}
		La méthode \textit{serialize} écrit le code Java, avant d'appeler cette méthode il est donc nécessaire de << remplir >> l'objet avec les informations nécessaires via les autres méthodes.
		
		\subsection{Le moteur de template : freemarker}
		\begin{wrapfigure}{l}{3cm}
			\includegraphics[width=3cm]{contents/images/FreeMarker.png}
		\end{wrapfigure}
		Les classes que je génère ont un gabarit\footnote{\'Egalement appelé \textit{template}} commun, seule l'implémentation des méthodes change, ainsi plutôt que réécrire systématiquement tout le corps de la classe, il parraissait intéressant d'avoir un moteur de \textit{template}, j'ai choisi FreeMarker.

		J'ai développé un fichier gabarit utilisant le format \textit{FreeMarker}, à l'interieur de celui-ci, j'utilise des objets. Lors de la sérialisation, je fournis les valeurs des objets concernés, l'adresse du fichier \textit{template}, le fichier \texttt{.java} est ensuite généré par FreeMarker.

		Deux fichiers de templates ont été nécessaire : un pour les stim, et un pour le \textit{GreenTTest}. En effet un precondstim et un stimscenario n'ont que peu de différence, il était donc possible de les regrouper avec le même template.

		Le template des stim contient une liste d'actions à exécuter dans la méthode d'exécution, et une liste d'alias nécessaire qui doivent être ajoutée dans une méthode adéquate. \\
		Le template des \textit{GreenTTest} lui contient une méthode permettant de remplir le rapport et une méthode qui créé l'\textit{Expected Behavior}.
		Les deux templates ont également une liste d'import qui est ajoutés automatiquement en début du fichier, ceux-ci sont disponibles dans l'annexe \ref{annexeTemplate} suivis de deux exemples de fichiers générés annexe \ref{annexeGeneration}.

	
		\begin{figure}[H]
		\centering
		\includegraphics[width=12cm]{contents/images/FreeMarkerSchema.png}
		\caption{Schéma de fonctionnement de FreeMarker}
		\end{figure}
		\begin{remarque}
		Cet exemple qui vient du site officiel de FreeMarker utilise des templates HTML, pour ma part j'ai créé des templates Java, mais le principe est le même.
		\end{remarque}
