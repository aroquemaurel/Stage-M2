\chapter{Bilans}
\putminitoc
%
Après ce stage d'un an, il est temps de dresser un bilan, du point de vue de Continental afin de voir ce que mon travail leur a apporté, mais également en quoi cette alternance d'un an a été bénéfique pour ma future carrière professionnelle.
%
\section{Bilan pour Continental}
Mon travail dans l'équipe de développement aura été intéressant pour l'entreprise, en partie grâce à ma connaissance de l'outil suite à mes stages précédents. En effet, avoir contribué au projet les années précédentes sur la conception et le développement de celui-ci m'a permis de d'être très rapidemment opérationnel, et notamment de revenir en septembre directement là où je m'étais arrêté, c'est-à-dire la conception du nouveau système de verdicts.

Nous avons pu livrer une première version de l'outil en janvier après deux ans de développement. Cette première version nous à permis de mettre en place le cœur de l'outil sur lequel j'ai pu ajouter de nouvelles fonctionnalités, et construire autour d'une base solide.

Comme présenté chapitre \ref{collab}, j'ai très rapidement mis en place l'adaptation de GreenT aux projets multi-c\oe{}urs, cette adaptation à permis à l'équipe de développement des projets Ford Panther Phase 2 d'utiliser l'outil. C'est ainsi que cette équipe a été capable de rédiger 270 tests, avec peu de support, principalement l'utilisation de la documentation, ces tests peuvent ensuite être exécutés sur table de tests en 2h30. L'équipe peut donc effectuer des tests systématiques à l'ensemble des versions logicielles de leur projets de manière rapide, et grâce à l'outil il a déjà été possible de trouver une dizaine de bogues.

Le projet n'est pas encore terminé, en effet il reste encore beaucoup de fonctionnalités utiles, du support et de la maintenance à effectuer, je vais notamment continuer le développement de la synchronisation aux mois de Juillet et Août, tout en effectuant la transmission de connaissance. En effet, dès septembre, Alain \bsc{Fernandez} sera en charge de maintenir le projet et devra donc avoir toutes les cartes en mains pour être le plus efficace possible.

De plus, grâce à la synchronisation, on peut espérer que l'outil sera ensuite utilisé sur les projets Renault, ce qui leur fera gagner du temps, mais demandera plus de support.

%
%\newpage
\section{Bilan personnel}
Cette expérience en entreprise m'a beaucoup apporté, tout d'abord d'un point de vue technique, j'ai acquis de l'expérience en conception logicielle, grâce
à toutes nos réunions où nous réfléchissions à la meilleure approche possible. De plus lors de problèmes, le travail en équipe m'a permis d'avoir
une autre vision du problème et une autre manière de le résoudre !

J'ai également approfondi le domaine de l'automobile, notamment en étudiant le fonctionnement des ECU multi-cœur ou du fonctionnement du debugger avec l'utilisation du port JTag. J'ai pu aussi apprendre à utiliser l'analyseur logique de Lauterbach, le \textit{powerprobe}, et des notions d'électronique m'ont été utiles afin de ne pas endommager le matériel. J'aurai ainsi pu voir que l'informatique ne peut se suffire à elle même et nécessite la connaissance d'un domaine métier.

J'ai pu mettre en pratique les enseignements du Master DL, notamment ceux de M1 avec les modules MCPOO pour la conception UML, DCLL pour l'utilisation de Git, mais également le cours d'IVVQ nous ayant montrés les approches de validation et de vérification, celui de PCR pour les concepts que j'ai utilisés avec Thrift et enfin le cours d'IS pour la partie sur les exigences. \footnote{Plus de détails sur les différents modules sont disponible dans les références annexe \ref{references}}. Pour certaines parties du projet, j'ai appris en autodidactie, notamment l'utilisation de Python, ce projet est donc particulièrement intéressant par le nombre de technologies et savoirde connaissances qu'il a nécessité.

%Contrairement à l'année précédente, j'ai pu travailler directement sur les tables de tests et pu ainsi découvrir des notions d'embarqués et d'automobile, et j'ai pu mieux appréhender le fonctionnement de l'ECU. 
%
Mais j'ai aussi acquis des connaissances humaines avec notamment le travail en équipe, communiquer sur nos avancements, de manière écrite ou orale, et être capable de synthétiser ses propositions de manière claire et concise. Étant le seul développeur de nouvelles fonctionnalités depuis janvier, j'ai également pu gagner en expérience sur l'analyse des besoins et la synthétisation de ceux-ci.

J'ai également eu le plaisir de revenir dans une multinationale, avec des collègues souhaitant toujours transmettre leurs connaissances et leur expérience, notamment dans le monde de l'automobile et de l'embarqué. 

Un bilan très positif donc, qui m'a réconforté dans mon projet professionnel : le développement logiciel, d'outil ou de client lourd dans l'industrie, et de me diriger vers de l'expertise technique. Afin de pouvoir suivre ce projet, j'ai choisi d'aller travailler dans une société de service, Extia, qui me permettra ensuite de travailler au sein d'un client intéressant.
