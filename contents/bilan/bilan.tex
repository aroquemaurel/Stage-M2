\chapter{Bilans}
\putminitoc

Après ce stage de quatre mois, il est temps de dresser un bilan, du point de vue de Continental, ce que mon travail leur a apporté, mais également en quoi ce stage a été bénéfique pour ma future carrière professionnelle.

\section{Bilan pour Continental}
Mon travail dans l'équipe de développement aura été intéressant pour l'entreprise, en partie grâce à ma connaissance de la plateforme suite à mon stage de licence. En effet, avoir contribué au projet l'année précédente sur la conception de celui-ci m'a permis de rapidement commencer le travail et de corriger des bugs répartis dans différents modules. De plus, revenir huit mois plus tard sur ce projet m'a permis d'appréhender le logiciel de manière plus globale et j'ai ainsi pu soulever des problèmes que nous n'avions pas vu lors de la conception.

Grâce à mes connaissances de l'architecture j'ai aidé Benjamin \bsc{Guerin} -- le troisième membre de l'équipe arrivant sur le projet -- j'ai ainsi pu lui donner des explications et des conseils, pendant que lui apportait un regard neuf à l'existant.

Comme présenté chapitre \ref{collab}, mon travail aura été directement utile à l'équipe de développement, et au groupe TAS. En effet, j'ai développé deux nouvelles fonctionnalités attendues par le client mais j'ai également corrigé des bugs. Ces corrections de bugs ainsi que les nouvelles fonctionnalités nous permettent maintenant d'exécuter les stimulations ainsi que l'analyse complète sur la dernière version du projet Ford : il est maintenant possible d'avoir les résultats de 87 tests en une heure.

Le projet n'est pas terminé, et je n'ai pas pu effectuer toutes les fonctionnalités auxquels nous avions pensés tel que l'utilisation de GreenT avec d'autres fichiers d'entrées que le Walkthrough. Ceci est principalement due à de mauvaises estimations, notamment en raison de la maintenance demandant plus de temps que prévu. Cependant, je vais continuer ce projet dès septembre en contrat de professionnalisation pendant un an et pourrais ainsi finaliser la plateforme.

\newpage
\section{Bilan personnel}
Cette expérience en entreprise m'a beaucoup apporté, tout d'abord d'un point de vue technique, j'ai acquis de l'expérience en conception logicielle, grâce
à toutes nos réunions où nous réfléchissions à la meilleure approche possible. De plus lors de problèmes, les propositions des autres m'ont permis d'avoir
une autre vision du problème et une autre manière de le résoudre !

Contrairement à l'année précédente, j'ai pu travailler directement sur les tables de tests et pu ainsi découvrir des notions d'embarqués et d'automobile, et j'ai pu mieux appréhender le fonctionnement de l'ECU. 

Mais j'ai aussi acquis des connaissances humaines avec notamment le travail en équipe, communiquer sur nos avancements, de manière écrite ou orale, et être capable de synthétiser ses
propositions ou de réussir à poser un problème rapidement tout en se faisant comprendre.

J'ai également eu le plaisir de revenir dans une multinationale, avec des collègues souhaitant toujours transmettre leurs connaissances et leur expériences, notamment dans le monde de l'automobile et de l'embarqué. 

Un bilan très positif, qui m'a réconforté dans mon projet professionnel : ma continuation en M2 Développement Logiciel, en alternance chez Continental.
