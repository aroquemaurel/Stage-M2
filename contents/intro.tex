\chapter*{Introduction}
	\addcontentsline{toc}{chapter}{Introduction} 


Dans le cadre de ma formation en première année de Master spécialité Développement Logiciel à l'université Toulouse III – Paul Sabatier, j'ai eu la possibilité d'effectuer un stage.


Attiré par le monde de l'entreprise et désireux de gagner en expérience, j'ai eu l'opportunité de continuer un projet commencé l'année précédente lors de mon stage de fin de Licence dans l'entreprise Continental Automotive : le développement d'une plateforme de tests de logiciels embarqués.

Ce projet a pour but d'aider une équipe de Continental, celle-ci est en charge de l'intégration d'un plugin, fourni sous forme binaire, que l'équipe doit intégrer dans le logiciel d'un calculateur de contrôle moteur. Pour cela, une plateforme permettant d'effectuer des tests automatisés est en développement.

Ayant connu les prémices de ce projet, et afin d'avoir un aperçu de celui-ci sur la longueur, allant de sa conception jusqu'à sa mise en production, le sujet du stage était particulièrement intéressant. En outre, celui-ci est en parfaite adéquation avec mon projet professionnel, et ma poursuite en M2 Développement Logiciel.

C'est au sein de l'équipe \textit{Test \& Automation Service} que j'ai effectué mon stage d'une durée de quatre mois, je vais ainsi vous présenter en quoi le développement de cet outil est nécessaire à l'équipe en charge des tests de ce plugin. Dans une première partie nous présenterons l'entreprise Continental et plus particulièrement l'équipe Tests \& Automation Service(chapitres \ref{chapConti} et \ref{chapOrganization}). Nous aborderons ensuite le problème que posent actuellement les tests de ce plugin(chapitre \ref{chapPb}), avant de présenter la solution qui est en cours de développement(chapitre \ref{chapGreent}) et comment j'ai contribué à ce projet(chapitre \ref{collab}). 
