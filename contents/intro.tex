\chapter*{Introduction}
	\addcontentsline{toc}{chapter}{Introduction} 
Dans le cadre de ma formation en seconde année de Master spécialité Développement Logiciel à l'université Toulouse III – Paul Sabatier, j'ai eu la possibilité d'effectuer un contrat de professionnalisation en alternance d'une durée d'un an.

Attiré par le monde de l'entreprise et désireux de gagner en expérience, j'ai pu continuer un projet commencé précédemment lors de mes stages de licence et de M1, dans l'entreprise Continental Automotive : le développement d'un outil de tests de logiciels embarqués.

Ce projet a pour but d'aider une équipe de Continental. Celle-ci à des difficulté au niveau de l'intégration d'un plugin dans un logiciel de contrôle moteur. Celui-ci possède des milliers de variables interfaces et est donc compliqué à tester. Afin d'aider cette équipe, un outil permettant d'effectuer des tests automatiques est en développement : \textit{GreenT}.

Dans un premier temps, afin que ce projet puisse disposer d'une première version de production, il a été nécessaire de refondre le système de prononciation des verdicts des tests.\newline
Dans un second temps, j'ai effectué du support, de la formation et de la maintenance sur le projet, tout en continuant le développement afin d'ajouter de nouvellse fonctionnalités.\newline
Ayant connu les prémices de ce projet, et afin d'avoir un aperçu de celui-ci sur la durée, allant de sa conception jusqu'à son exploitation, le sujet du stage était particulièrement intéressant. En outre, celui-ci est en parfaite adéquation avec mon projet professionnel, le développement logiciel au sein d'industrie. En effet, ce projet est au cœur du problème d'ingénieur logiciel, par la problématique que celui-ci essaye de régler :  la manière dont nous pouvons tester un logiciel. 


J'ai travaillé au sein de l'équipe \textit{Test \& Automation Service}, je vais ainsi vous présenter en quoi le développement de cet outil est nécessaire à l'équipe en charge des tests de ce plugin.\\ Dans une première partie nous présenterons le contexte dans lequel j'ai travaillé, avec l'entreprise Continental et plus particulièrement l'équipe \textit{Tests \& Automation Service}(chapitre \ref{chapConti}), puis nous verrons le problème que posent actuellement les tests de ce plugin(chapitre \ref{chapPb}), nous aborderons ensuite la manière dont nous nous sommes organisés pour le développement (chapitre \ref{chapOrganization}) avant de présenter la solution qui est en cours de développement(chapitre \ref{chapGreent}) et comment j'ai contribué à ce projet(chapitre \ref{collab}). 
