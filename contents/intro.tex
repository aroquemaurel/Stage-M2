\chapter*{Introduction}
	\addcontentsline{toc}{chapter}{Introduction} 
Dans le cadre de ma formation en première année de Master spécialité Développement Logiciel à l'université Toulouse III – Paul Sabatier, j'ai eu la possibilité d'effectuer un stage d'une durée de quatre mois.

Attiré par le monde de l'entreprise et désireux de gagner en expérience, j'ai eu l'opportunité de continuer un projet commencé l'année précédente dans l'entreprise Continental Automotive : le développement d'une plateforme de tests de logiciels embarqués.

Ce projet a pour but d'aider une équipe de Continental. Celle-ci est face à un problème : l'intégration d'un plugin dans un logiciel de contrôle moteur. Celui-ci possède des centaines de variables interfaces et est donc compliqué à tester. Afin d'aider cette équipe, une plateforme permettant d'effectuer des tests automatiques est en développement : \textit{GreenT}.

Ce projet est encore aujourd'hui en cours de développement, et pour pouvoir disposer d'une première version de production, il est nécessaire d'implémenter de nouvelles fonctionnalités et de corriger des bugs. Ayant connu les prémices de ce projet, et afin d'avoir un aperçu de celui-ci sur la durée, allant de sa conception jusqu'à sa mise en production, le sujet du stage était particulièrement intéressant. En outre, celui-ci est en parfaite adéquation avec mon projet professionnel, et ma poursuite en M2 Développement Logiciel. En effet, ce projet est au cœur du problème d'ingénieur logiciel, par la problématique que celui-ci essaye de régler, notamment la manière dont nous pouvons tester un logiciel. 


%% TODO Descendre le niveau d'abstraction d'annonce du plan
C'est au sein de l'équipe \textit{Test \& Automation Service} que j'ai effectué mon stage d'une durée de quatre mois, je vais ainsi vous présenter en quoi le développement de cet outil est nécessaire à l'équipe en charge des tests de ce plugin. Dans une première partie nous présenterons l'entreprise Continental et plus particulièrement l'équipe \textit{Tests \& Automation Service}(chapitre \ref{chapConti}), puis nous verrons de quelle manière nous nous sommes organisés pour le développement (chapitre \ref{chapOrganization}). Nous aborderons ensuite le problème que posent actuellement les tests de ce plugin(chapitre \ref{chapPb}), avant de présenter la solution qui est en cours de développement(chapitre \ref{chapGreent}) et comment j'ai contribué à ce projet(chapitre \ref{collab}). 
