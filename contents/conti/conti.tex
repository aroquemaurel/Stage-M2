\chapter{Continental}\label{chapConti}
\begin{wrapfigure}{r}{0.60\textwidth}
\vspace{-25px}
\hspace{-30px}
\begin{minipage}{0.67\textwidth}
\minitoc
\end{minipage}
\end{wrapfigure}
Comme nous l'avons vu dans l'introduction, j'effectue mon stage au sein de l'entreprise Continental, une entreprise ayant pris une très grande importance dans le monde de l'automobile. 

Nous allons voir dans quel contexte opère cette société, et plus particulièrement l'équipe dans laquelle j'ai travaillé : l'équipe \textit{Tests Automated Service}.
	\section{Organisation de l'entreprise}
		\subsection{Continental AG}

Continental AG est une entreprise allemande fondée en 1871 dont le siège se situe à Hanovre. Il s'agit d'une Société Anonyme (SA) dont le président du comité de
direction est le Dr. Elmar \bsc{Degenhart} depuis le 12 août 2009. Elle est structurée autour de deux grands groupes : le groupe Rubber et le groupe Automotive.
	 
		 \begin{figure}[H]
		 	\centering
		 	\includegraphics[width=12cm]{contents/images/caConti.jpg}
		 	\caption{Chiffre d'affaire et nombre d'employés (Année 2011)}
		 	\label{fig:caConti}
		 \end{figure}

		 En 2013, l'entreprise comptait plus de $177000$ employés dans le monde\footnote{Cf figre \ref{fig:caConti}} répartis dans 269 sites et 46 pays différents\footnote{Cf figure \ref{fig:repartitionConti}}. Avec un chiffre d'affaire de 30.5 milliards d'euros au total, Continental est le numéro un du marché de production de pneus en Allemagne et est également un important équipementier automobile.
		 \begin{figure}[H]
		 	\centering
		 	\includegraphics[width=13cm]{contents/images/repartitionConti.jpg}
		 	\caption{Répartition du groupe continental dans le monde}
		 	\label{fig:repartitionConti}
		 \end{figure}		 

		\subsection{Histoire de l'entreprise}
		Continental est fondée en 1871 comme société anonyme sous le nom de <<Continental-Caoutchouc-und Gutta-Percha Compagnie>> par neuf banquiers et industriels de Hanovre (Allemagne).

		Continental dépose l'emblème du cheval représenté sur la figure \ref{fig:logo}, comme marque de fabrique à l'Office impérial des brevets de Hanovre en octobre 1882. Ce logo est aujourd'hui encore protégé en tant que marque distinctive.
		\begin{figure}[H]
			\centering
			\includegraphics[width=2cm]{contents/images/logo.jpg}
			\caption{Logo de Continental}
			\label{fig:logo}
		\end{figure}

		Le fabricant de pneus allemand débute son expansion à l'international en tant que sous-traitant automobile international en 1979, expansion qu'il n'a cessé de poursuivre depuis.
		
Entre 1979 et 1985, Continental procède à plusieurs rachats qui permettent son essor en Europe, celui des activités pneumatiques européennes de l'américain \textit{Uniroyal Inc.} et celui de l'autrichien \textit{Semperit}.

En 1995 est créée la division << Automotive Systems >> pour intensifier les activités << systèmes >> de son industrie automobile.

La fin des années 1990 marque l'implantation de Continental en Amérique latine et en Europe de l'Est.

En 2001, pour renforcer sa position sur les marchés américain et asiatique, l'entreprise fait l'acquisition du spécialiste international de l'électronique \textit{Temic}, qui dispose de sites de production en Amérique et en Asie. La même année, la compagnie reprend la majorité des parts de deux entreprises japonaises productrices de composants d'actionnement des freins et de freins à disques. 

En 2004, le plus grand spécialiste mondial de la technologie du caoutchouc et des plastiques naît de la fusion entre \textit{Phoenix AG} et \textit{Conti'Tech}.

Enfin en juillet 2007, Continental réalise sa plus grosse opération en rachetant le fournisseur automobile \textit{Siemens VDO Automotive}. Ce rachat a permis à l'entreprise de multiplier son chiffre d'affaire par deux, passant ainsi de 13 milliards d'euros à plus de 30 milliards d'euros (chiffre de 2011).
		
		\subsection{Activités des différentes branches}
		\begin{figure}[H]
			\centering
			\includegraphics[width=18cm]{contents/images/structureConti.jpg}
			\caption{Structure de continental}
			\label{fig:structConti}
		\end{figure}

		Comme on peut le voir sur la figure \ref{fig:structConti}, Continental est composée de deux groupes et de six divisions. Ces dernières se chargent de développer et produire des équipements répondant aux besoins des clients. Pour cela elles sont composées de Business Units qui ont chacune une activité bien particulière dans leur domaine de compétence. 

Durant mon stage, je travaillais au sein de la division \textit{powertrain}. Elle s'occupe essentiellement du contrôle moteur, au niveau logiciel et materiel avec l'ECU\footnote{Electronic Control Unit} et de la mise au point des systèmes diesel et essence. Par exemple, la Business Unit \textit{Engine Systems} est chargée de produire les équipements nécessaires au contrôle moteur tels que des calculateurs ou des injecteurs.

	\section{Le contexte de l'équipe Test Automated Service}
% 		\subsection{L'équipe}
% 		J'ai travaillé dans l'équipe en charge de la vérification et de la validation des logiciels.\footnote{Plus précisément dans le service P-ES-E-SYS-ETV-V.\newline P: Powertrain\newline ES-E-SYS: Engine Systems\newline ETV: Engineering Tool and Verification\newline V: Vérification.} dirigée par Stéphane \bsc{Bride}. Cette équipe est en charge du développement, de la configuration et de l'exécution de scripts de tests de non-régression automatique\footnote{Aussi appelés FaST : Functions and Software Testing} sur bancs HIL\footnote{Hardware in the Loop} avant la livraison des projets.
		
 		\subsection{Le besoin} \label{besoinTests}
 		Le calculateur du contrôle moteur d'une voiture est un dispositif très important et à haut risque, puisqu'une défaillance peut provoquer la mort de plusieurs personnes. Ainsi, le test est indispensable dans ce domaine, et doit être robuste. 

L'automatisation des tests est rendue nécessaire pour deux raisons. Tout d'abord, un logiciel ne peut comporter aucun bug, cependant les erreurs et bugs critiques liés à l'inattention peuvent être grandement réduits grâce à ce processus. De plus, au vue du très grand nombre de cas à tester pour un contrôle moteur, une opération manuelle serait impensable. 

C'est dans ce contexte que l'équipe \textit{Test Automated Service} intervient, elle doit fournir des outils aux développeurs afin de vérifier facilement et correctement leur travail, particulièrement pour des tests de régression, bien que l'outil sur lequel je travaille soit à destination de tests d'intégration.
