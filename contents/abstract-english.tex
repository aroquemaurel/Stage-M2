\textbf{Keywords}: Automotive, Tool, ECU, Control engine, Tests, Workbench, HiL, Debugger, Java, Python

During my second year of Master in computer science at Toulouse III University – Paul Sabatier,
I had the opportunity to do an internship. I chose to work in continuity of my 1st year, the development
of a tool for embedded tests, in Continental, an automotive company.

Continental is a leading German automotive manufacturing company who has 174 000 employees
all around the world. The company is specializing in tires, brake systems, automotive safety
and powertrain. I worked in the team that was in charge of software tests, with the development
and configuration of scripts. Those scripts are for regressions tests, and in my case, integration tests.

Three years before this internship, a need was been expressed: being able to quickly and efficiently
test the integration of a plugin, meaning a piece of code in binary. This plugin is for an
engine control software. The mission of the team Tests \& Automation Service is helping the proper
integration of this plugin with Continental software. For this purpose, the development of a test
tool is required.

The development of this tool, called GreenT, began three years ago. GreenT needs bugs corrections
and features development for production deployment. So, I worked in this context during
one year, and in january, the release 1.0 was deployed.

The goal of this tool was to provide the most exhaustive and effective tests possible. For
this purpose, the customer provides a specification file, called Walkthrough, wich contains the specification
of all plugin's variables. Testers will add columns in this file for test specification. After
this, GreenT will be able to parse and generate tests case that will be executed remotely on one
or several workbenches. A workbench can simulate a vehicle environment around an ECU to check
reactions.

I developed some features in this project: the improvement of reports generation, which is required for delivery, and systems 
for opening GreenT to most of projects. In addition, since january, I support users which writing tests.

So my work during this internship was beneficial for the company, my knowledge about GreenT
allowing me to help fixing bugs, and I was able to develop some new features, and help to delivery the first operational version of the tool.
It was also beneficial for me, from a technical point of view by designing solutions for particular
problems, and from a human point of view by improving my teamwork skills thanks to the meetings
we did to talk about the development.

This project is not completed yet, it required maintenance and development of new features. However, this two years of devlopment have shown me a complete software lifecycle, from needs analyze to exploitation of tool. It will be an asset for my future professional career.