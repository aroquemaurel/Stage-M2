\chapter{Ma collaboration au projet}\label{collab}
\putminitoc
%Après avoir défini plus en détails les besoins de notre plateforme et son fonctionnement général, nous allons maintenant voir en détail de quelle manière j'ai contribué à ce projet. En parallèle de la maintenance de notre plateforme, j'ai développé deux nouvelles fonctionnalités. Ces deux fonctionnalités ayant un rapport direct avec la notion de << calibration >>, nous allons tout d'abord définir celles-ci avant de voir en détail mon développement et la maintenance que j'ai effectué.

\subsection{La difficulté de productions de rapports}

%Nous avons eu du mal à produire des résultats de tests fiables et efficaces : il a fallut adapter notre "analyzer" produisant ces résultats afin qu'il fonctionne dans tous les cas possible, ceci en se servant du concept des récurrences de calcul, et des variables d'entrée et de sortie d'un calcul. Rapide présentation de la sérialisation au format excel
\subsection{L'arrivée des projets multi-core}
%Présentation générale du mono-core vs multi-core, en quoi c'était nécessaire rapidement(nouveaux projets)
\subsection{Généralisation de l'outil à d'autres projets}
%Création d'un test plan "générque" permettant d'utiliser autre chose qu'un fichier Walkthrough : ouverture vers les projets Renault
\subsection{La difficulté de synchronisation des traces}
%Cette partie n'a été qu'étudiée pour le moment et n'est pas encore développée : cela concerne la possibilité de synchroniser le temps de deux traces différentes. (simulateur + debugger). Ce besoin est nécessaire aux projets Renault. Je pense présenter la solution, même si celle-ci n'aura pas nécessairement été développée lors de la rédaction du rapport. 
\subsection{Le support et la maintenance}
%Présentation rapide de notre utilisateur, de la quantité de tests à effectuer, de l'utilité de l'outil, du travail que j'ai du faire en terme de formation, de réponses aux besoins, corrections, et les problèmes que j'ai rencontré. 